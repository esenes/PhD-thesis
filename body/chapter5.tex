\chapter[Conclusions]{Conclusions}

Beam-driven novel acceleration schemes exploit various mechanisms to transfer the energy stored in a drive beam to accelerate a witness beam. The common denominator of these schemes is the large intensity difference between both beams, allowing the witness beam to be accelerated to a higher energy than that of the drive beam. Chapter~\ref{chapter:copropagating_beams} described a beam-diagnostics technique for a witness beam with bunches much shorter than those of the drive beam. This technique could be applied in the AWAKE experiment where a proton driver beam  stores a large amount of energy used to accelerate an electron witness beam.

A proton-based electron-beam acceleration was demonstrated during the so-called AWAKE Run 1. The ultimate future goal of the AWAKE experiment is the production of beams for particle-physics research, requiring very precise control of beam energy and brilliance and repeatability of operational conditions. The concept is often referred in the the plasma-acceleration community as "moving from acceleration to accelerators". 

For AWAKE Run 2, which aims to demonstrate emittance preservation of the accelerated beam, correct pointing of the electron beam in the plasma is crucial to control the acceleration process. The work documented in this thesis shows that this is possible but it requires a Beam Position Monitor working at frequencies of the order of tens of GHz.

This thesis work explored the possibility to build a high-frequency Beam Position Monitor based on the emission of ChDR from dielectric inserts in the beampipe. The produced signal can then be used to reconstruct the beam position. Beyond PWFA and beam-driven dielectric accelerators, the technique could be applied in any accelerator using short bunches, e.g. Free-Electron Lasers.

Chapter~\ref{chapter:CLEAR_test} describes using the ChDR for beam position monitoring together with the simulations and the test results of a prototype monitor with PTFE radiators at the CLEAR facility at CERN. It was experimentally demonstrated that a beam position monitoring system working in the Ka-band can be realised. Due to its proof-of-principle nature, the prototype performance was limited and the produced radiation was detected in air using already available components. Nevertheless, those first tests proved that measuring the position of both single bunches and bunch trains is possible. The device was the first of its kind to integrate the radiators in the beampipe. Electromagnetic simulation demonstrated the importance of the diffraction radiation produced by the beampipe discontinuities.

A second test campaign was carried out with an improved narrow-band detection system to study in detail the prototype's response to beam position and charge variations. The ChDR emission was measured with a beam charge as low as 2~pC, despite the large attenuation of the transmission line. The results showed that ChDR is a valid tool for particle-beam diagnostics. Furthermore, the conducted experiments allowed experience with RF Schottky diode detectors and their use in the transient regime to be gained. 

To use the ChDR BPM technology in an operational accelerator, a vacuum-tight monitor has to be developed. A preliminary design and simulations are presented in Chapter~\ref{chap:new_design}. Further steps required to fully develop a high-frequency BPM technology for short-bunch application are described in Section~\ref{sec:fd_chBPM}. For the specific case of AWAKE, where two particle beams are used, a BPM system architecture is proposed in Section~\ref{sec:fd_AWAKE}. The installation of a vacuum-compatible prototype is foreseen in 2021. 


