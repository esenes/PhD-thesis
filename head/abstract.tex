% \chapter*{Abstract}
\begin{abstract}

Novel acceleration technologies promise a large improvement in particle accelerator performance, but pose a number of technical challenges due to the use of several beams and the beam parameters involved. To exploit these new technologies, these technical challenges need to be addressed. Innovative beam instrumentation is to be devised, to allow these acceleration experiments to become operational accelerators.

The AWAKE experiment at CERN aims to develop proton beam-driven plasma wakefield acceleration, with the aim of producing high brightness and high energy particle beams for particle physics research. At AWAKE, plasma wakefields are excited by means of a 400~GeV proton beam driver, and used to accelerate an electron witness beam. The plasma is formed by ionising a Rubidium gas with a terawatt laser pulse. The laser, electron and proton beams co-propagate in the same beampipe for metres before entering the plasma. The electron beam diagnostic is obfuscated by the presence of the more intense proton beam. Consequently, the electron beam position cannot be measured in the presence of the proton beam. To drive the acceleration efficiently, a precise positioning of the three beams is crucial. Therefore, a technique to measure the electron beam position in the presence of the stronger proton beam has to be studied. 

This work addresses the beam position measurement when more than one beam is present in the beampipe. For the case of AWAKE, a technique to measure the electron-beam position exploiting the bunch-length difference with the proton beam is described. It is shown that the electron-position measurement can be carried out, provided that the detection frequency is sufficiently high. In the second part, a novel beam-position-measurement device, capable of working in the required frequency regime, is developed. Such a device is based on the emission of Cherenkov Diffraction Radiation from dielectric inserts in the beampipe. Electromagnetic simulations of the device are shown, together with the results of experimental tests on a prototype. Further developments to produce an operational instrument are discussed. The potential applications of this technology are not only in plasma-acceleration schemes, but also in any accelerator that uses short bunches, e.g. Free Electron Lasers. 

\end{abstract}